% Created 2023-12-14 Thu 09:59
% Intended LaTeX compiler: pdflatex
\documentclass[11pt]{article}
\usepackage[utf8]{inputenc}
\usepackage[T1]{fontenc}
\usepackage{graphicx}
\usepackage{grffile}
\usepackage{longtable}
\usepackage{wrapfig}
\usepackage{rotating}
\usepackage[normalem]{ulem}
\usepackage{amsmath}
\usepackage{textcomp}
\usepackage{amssymb}
\usepackage{capt-of}
\usepackage{hyperref}
\usepackage{bm}
\usepackage[margin=1in]{geometry}
\usepackage{amsmath}
\author{Jonathan Babyn}
\date{\today}
\title{}
\hypersetup{
 pdfauthor={Jonathan Babyn},
 pdftitle={},
 pdfkeywords={},
 pdfsubject={},
 pdfcreator={Emacs 27.1 (Org mode 9.3)}, 
 pdflang={English}}
\begin{document}



\section{Introduction}
\label{sec:org4b7a35b}

  \texttt{varCKMR} is a pseudo R package to showcase a method of estimating the variance in
the distribution of number of offspring. It does this using within birth-cohort 
sibling comparisons. 

It contains an individual based simulation that installs as the \texttt{R} package \texttt{varCKMR} along with 
some helper functions.
The \texttt{TMB} models used in the analysis are in the \texttt{models} folder with the \texttt{ssmodelAdj.cpp}
being the model using within-cohort sibling comparisons and the \texttt{ssmodelNoSC.cpp} model 
omitting them.

The \texttt{scripts} folder contains the scripts to recreate the simulations used in the analysis. \texttt{populations}
is where the \texttt{populations} used in the analysis would be kept althought they are ommitted here for space reasons. 
\texttt{sims} contains all the samples from each population but are again ommitted for space reasons. 

The org file \texttt{varCKMR.org} contains all the files to make all the files in this folder.

\section{A note on \(\sigma^2_{tot}\)}
\label{sec:org8beaf97}

The code here uses a more complicated derivation  to break up \(\sigma^2_{tot}\) (and \(Var(R)\)) then what is given in the
paper:

\begin{multline}
\label{muteqlaw2}
\sigma^2_{tot} = Var(X) = \sum_{a=0}^A Var(X|D = a)\Delta_a + \sum_{a=0}^A E[X| D = a]^2(1-\Delta_a)\Delta_a- \\ 
 2\sum_{a=1}^A\sum_{b=0}^{a-1} E[X|D = a] \Delta_a E[X|D = b] \Delta_b.
\end{multline}

This form assumes that X is also mutually exclusive (which age of death and age of parent are) but if that is the case then the form above and the form
given in the paper will agree.

\begin{verbatim}
##See README
##here is a small example using the sim parameters from one of the populations

surv = c(0.215,0.280)
death_probs = c(1-surv[1],surv[1]*(1-surv[2]),surv[1]*surv[2])
fecundity = 2*c(0.380,1.518,3.416)
theta = 0.1

EXD = cumsum(fecundity)
V1 = fecundity+fecundity^2/theta
varXD = cumsum(V1)


oldform <- function(varXD,EXD,Delta){
  firstsum <- sum(varXD*Delta)
  secondsum <- sum(EXD^2*(1-Delta)*(Delta))

  thirdsum <- 0
  for(i in 2:length(EXD)){
    for(j in 1:(i-1)){
      thirdsum = thirdsum + EXD[i]*Delta[i]*EXD[j]*Delta[j]
    }
  }

  total <- firstsum + secondsum -2*thirdsum
  total


}

newform <- function(varXD,EXD,Delta){
  mu_tot = sum(EXD*Delta)
  firstsum = sum(varXD*Delta)
  secondsum = sum(EXD^2*Delta)
  total = firstsum+secondsum-mu_tot^2
  total
}

oldv = oldform(varXD,EXD,death_probs)
newv = newform(varXD,EXD,death_probs)
print(paste0("old: ",round(oldv,3)," new: ", round(newv,3)))
\end{verbatim}

\begin{verbatim}
old: 61.673 new: 61.673
\end{verbatim}
\end{document}